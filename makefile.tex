%File Name:b.tex
%Created Time: 2019-05-04 09:57:09 week:6
%Author: hanhj
%Mail: hanhj@zx-jy.com 
\documentclass{article}
\usepackage{CJK}
\usepackage{graphicx}
\usepackage{indentfirst}%产生首行缩进
\usepackage{makeidx}
\makeindex
\begin{document}
\begin{CJK}{UTF8}{gbsn}
\title{Makefile文件使用简要说明}
\author{hanhj}
\maketitle
	\section*{前言}
	Makefile是一种脚本文件,被make命令所使用,用来自动执行一些命令,从而简化人们的工作。最常见的用途是用来编译程序,比如编译c或cpp文件。当然也可以用来做其他一些自动化工作,比如编译tex文件,生成pdf文档。编译doxygen文件,生成帮助文档等等。
	\par
	所以可以理解makefile文件就是一个用来实现自动化工作的脚本文件,与其他比如sh脚本文件一样的,只不过sh文件被bash所解释执行,而makefile文件被make所解释执行,二者有相同的地方,也有一些不同的地方,比如关于变量定义,使用都很相似,而在流程控制方面有些不同。此外,makefile也有一些自己的处理函数。\par
	查看make和makefile的在线帮助可以用man make和info make来获得。\par
	makefile中的注释是用\#开头。
	\section{基本命令}
	\subsection{第一个makefile}
	\begin{verbatim}
	test:test.c
	    gcc test.c -o test
	\end{verbatim}
	\par
	假如我们有test.c文件,想把它编译成test程序。在命令行中,我们输入如下命令:gcc test.c,就可以了。但是,如果每次修改完test.c我们都要敲如上的字符,觉得太麻烦了,这时makefile就可以登场了,敲入如上代码,然后保存到makefile(或Makefile)文件中,这时,我们每次修改完test.c文件,直接敲make就可以得到test程序了。


	注意第二行,前面需要用Tab键缩进。
	\par
		解释一下,第一行test:test.c,前面的test表示要生成的目标,后面的test.c表示为了生成这个目标所要依赖的文件,就是说为了生成test这个程序,需要test.c这个文件,这句话表示的就是这种依赖关系。  
	\par
	第二行gcc test.c -o test表示的是为了生成这个目标所要执行的动作,其实就是我们在命令行中所敲入的命令。不过需要注意的是在makefile文件中,描述这种动作的语句需要在前面用Tab键缩进。\\
	   \framebox{\shortstack[l]{ 
		目标:依赖的文件\\
		\indent\indent为生成目标所要执行的命令
		}
		}\\

	Makefile的基本语法就是这样。
	\subsection{变量}
	在makefile中使用变量可以替换字符,提高效率与灵活性。
	\subsubsection{变量定义}
	\begin{verbatim}
		var=xxx
	\end{verbatim}
	\par
		将xxx赋值给变量var。使用的时候,用\$(var)或\$\{var\},当make命令遇到该符号时,将其用定义时的字符串展开。即用xxx来代替\$(var)。
	\par
		var=xxx 是一种\textbf{延后展开方式}。即如果xxx中还包括变量,则make会在扫描完所有makefile文件之后,明确该变量的值后再赋值给var。此外其他几种赋值方式:\\
		\begin{table}[!htp]
		\begin{tabular}[t]{l|l}
		\hline
		var:=xxx & 立即展开方式。即如果xxx中还包括变量,则在此处立即展开该变量。\\
		\hline
		var+=xxx & 加方式。在var后面添加xxx变量\\
		\hline
		var?=xxx & 预定义方式。即如果var这个变量如果预先定义过了就用预先定义的值,否则用xxx\\
		\hline
		\end{tabular}
			\caption{变量赋值}
		\end{table}	
		\newline
		举例:
	\begin{verbatim}
		var=$(name)
		name=john
		all:
		    @echo $(var)
		make之后结果为john。因为name在后面赋值了。虽然name的赋值语句在var赋值语句之后,但是make依然会得到正确结果。

		var:=$(name)
		name=john
		all:
		    @echo $(var)
		make之后的结果为空。因为make在看到:=的赋值语句后,立即对name变量展开,而此时name并没有赋值,所以var的值为空。

		CC?=gcc
		all:
		    @echo $(CC)
		make之后的结果是cc,而不是gcc。因为在linux中CC预定义为cc。

		var=merry
		var+=john
		all:
		    @echo $(var)
		make之后的结果是merry john 
	\end{verbatim}
	\subsubsection{变量的用途}
		变量可以用作表示目标文件,依赖文件,或者一些参数等等。
	\begin{verbatim}
		举例:
		用作依赖文件:
		objs=main.o test.o 
		test:$(objs)		#test依赖于objs
		    gcc $(objs) -o test

		用作目标文件:
		target=test
		$(target):test.o    #target 是目标文件
		    gcc test.o -o $(target)
		
		用作参数:
		cflags=-c -g		#cflags是编译c文件的参数,-c表示仅编译,不生成目标,-g 表示在目标文件中加入调试信息。	
		test.o:test.c
		    gcc $(cflags) test.c -o test.o
		test:test.o
		    gcc test.o -o test 
	\end{verbatim}		
	\subsubsection{常用的makefile中的内部变量}
	为了方便,在makefile中定义了一些内部变量,可以方便的使用以节省敲字符。\\
	\noindent\textbf{自动化变量:}\\
	\begin{table}[!htp]
	\begin{tabular}{l|l}
		\hline
		\$@ & 代表目标文件\\
		\hline
		\$$<$ & 代表依赖关系中的首个依赖文件\\ 
		\hline	
		\$\^ & 代表依赖关系中的所有依赖文件\\
		\hline	
		\$? &	代表依赖关系中依赖文件中更新文件\\
		\hline	
	\end{tabular}
		\caption{自动化变量}
	\end{table}
			

	\noindent用上面的变量改写第一个makefile:\nolinebreak[2]
	\begin{verbatim}
		test:fun1.c test.c #假设依赖两个文件
		    gcc $^ -o $@
	\end{verbatim}
	\newpage
	\textbf{通配符:}\\makefile中的通配符*,?,[]一般用来匹配文件,与shell中的用法基本一致。\\
	\begin{table}[!htp]
	\begin{tabular}{l|l}
		\hline
		* & 匹配任意字符任意次\\
		\hline
		? & 匹配任意一个字符1次\\
		\hline
		[..] & 匹配[...]中出现的字符1次\\
		\hline
		除此以外,还有 &\\
		\hline
		\% & 匹配任意1到多个字符,常用于依赖规则书写\\
		\hline
	\end{tabular}
		\caption{通配符}
	\end{table}
	
	举例:\nolinebreak[2]
	\begin{verbatim}
		srcs=*.c	#匹配当前目录下的所有.c文件
		makefile=[Mm]akefile #匹配Makefile或makefile文件 
		main_c=main?.c #匹配mainx.c,x是任意字符。
	\end{verbatim}


	\textbf{默认的编译相关命令}\\
	\begin{table}[!htp]
	\begin{tabular}{l|l|l}
		\hline
		变量名 & 命令 & 默认文件后缀名\\
		\hline
		\$(CC) & cc, c程序编译器 & c\\
		\hline
		\$(CXX) & g++, c++程序编译器 & cpp,cc,C\\
		\hline
		\$(LD) & ld, 连接器 & o\\
		\hline
		\$(AS) & as,汇编程序编译器 & s,S\\
		\hline
		... & ... & ...\\
		\hline
	\end{tabular}
		\caption{默认的编译相关命令}
	\end{table}
	\subsection{语法}
	makefile中的语法包括依赖关系,命令等。\\
	最基本的语法就是开始所说的:\\
	\framebox{\shortstack[l]{
		目标:依赖的文件\\
		\indent\indent为生成目标所要执行的命令
	}}\\
	\subsubsection{显式依赖规则}
	就像开头的第一个makefile一样,我们在makefile中明显的指明目标与依赖文件之间的关系,并指明由依赖文件生成目标文件的命令。这种方式称为显式依赖。\par
	采用这种方式的好处是依赖关系非常直观,但是也存在输入字符太多,如果项目中的文件很多,则这种人工的方式显得比较低效。而且一旦文件名发生变动,需要手动的修改makefile文件,也比较费事。\par
	为此,我们可以创建一种规则,通过文件后缀名建立一种依赖关系。这样我们修改的就是项目中所包括的文件名,而不用修改依赖规则。下面是一个改进的makefile:\par
	\begin{verbatim}
		target=test
		objs=fun.o main.o
		%.o:%.c
		    $(CC) -c -g $< -o $@
		target:$(objs)
			gcc $^ -o $@ 
	\end{verbatim}
		这里,我们建立了一个从.c文件到.o文件的显式依赖。
	\subsubsection{隐式依赖规则}
	除了用显式规则建立依赖关系,对于makefile来说还有一种称为隐式依赖规则,即如果没有在makefile中明显的根据文件后缀名依赖关系,make命令会自动的根据一般的理解建立这种依赖关系。比如如下代码:\par
	\begin{verbatim}
		target=test
		objs=fun.o main.o
		target:$(objs)
			gcc $^ -o $@
	\end{verbatim}


		与上面一段代码所不同的是,这里没有建立.c到.o的依赖关系,但是依然能够正确执行,原因就是make命令自动建立了依赖关系。其实就是建立了如下的代码:\par
		\begin{verbatim}
			%.o:%.c
			    $(CC) -c $(CFLAGS) -o $@ $<
		\end{verbatim}


		CFLAGS是一个内部变量,用于在编译c程序的时候使用,需要在make命令的时候在命令行中给出,或者在makefile文件中给出。默认是空。这一点,我们在执行make命令的时候,可以通过其执行过程可以看出。\par
	
	\subsubsection{包含其他文件}
		include xxx\par
		makefile文件可以包括其他文件,包括makefile文件。在makefile文件中包括其他文件,有利于将各部分内容分开,有利于模块化。将经常变动的部分与不变部分分开,也有利于对项目的维护。\par
		常见的做法是将项目中全局的定义,全局的依赖规则定义在一个主文件中,其他makefile文件仅负责其本地有变化的内容。
	\subsubsection{搜索路径}
		\noindent VPATH=path\\
		\noindent vapth patten path 
		\par
		设置搜索路径的目的是为makefile文件中的目标文件或依赖文件在当前目录找不到的情况下,提供一个搜索的路径。\par
		注意,该路径并不提供在源文件中所包括的头文件的搜索路径,仅仅是为makefile中出现的文件服务的。\par
		举例:
		\begin{verbatim}
			VPATH=src:include #两个以上目录用:分开
			vpath %.c src #模式搜索,为.c文件提供搜索路径。
		\end{verbatim}
	\subsubsection{伪目标}
		\begin{verbatim}
		.PHONY:xxx
		xxx:
		    command
		\end{verbatim}\par
		伪目标不生成文件,但是往往是需要执行的动作。比如清除编译过程中的临时文件。\\
		举例:\par
		\begin{verbatim}
			target=test
			objs=fun.o main.o
			$(target):$(objs)
			    gcc $^ -o $@
			.PHONY:clean
			clean:
			    rm -f $(target) $(objs)
		\end{verbatim}
	\subsubsection{命令的前缀}
		在执行的命令前可以有-,@,+前缀表示不同的执行方式。
		\begin{itemize}
				\item - :表示即使命令执行出错,也继续执行下去。
					\par 比如
					\begin{verbatim}
						target=test
						objs=fun.o main.o
						$(target):$(objs)
						    gcc $^ -o $@
						.PHONY:clean
						clean:
						   -rm -f $(target) $(objs) #这里用-rm
					\end{verbatim}		
				\item @ :表示不显示执行的命令本身,而只显示执行结果。
					\begin{verbatim}
							target=test
							objs=fun.o main.o
							$(target):$(objs)
							    gcc $^ -o $@
							.PHONY:clean
							clean:
							    rm -f $(target) $(objs)
							    @echo rm  $(target) $(objs) #
					\end{verbatim}				
				\item + :不常用,表示即使make命令行中包括-n,-t,-p选项也要执行。
		\end{itemize}

	\subsection{makefile中常用函数}	
		makefile中的函数调用方法是:
			\begin{verbatim}
				$(fun,arg1,arg2...)
			\end{verbatim}
		\subsubsection{模式变量替换}
		\begin{verbatim}
			$(list_str:patten=replace)
			将list_str中匹配到patten的部分替换成replace。严格的说这个不算函数。
			比如:
			objs=func.o main.o
			depend_file=$(objs,%.o=%.d)
			结果depend_file为func.d main.d
		\end{verbatim}
		\subsubsection{subst变量替换函数}
		\begin{verbatim}
			$(subset,from,to,txt)
			将txt中的from的字符替换成to
			举例:
			objs=func.o main.o
			depend_file=$(subst,o,d,$(objs))
			结果depend_file为func.d main.d
		\end{verbatim}
		\subsubsection{patsubst模式变量替换函数}
		\begin{verbatim}
			$(patsubset,patten,replace,txt)
			将txt中匹配到patten的字符替换成replace。可以使用通配符。
			举例:
			objs=func.o main.o
			depend_file=$(patsubst,%.o,%.d,$(objs))
			结果depend_file为func.d main.d
		\end{verbatim}

	\subsection{make命令}
	\subsubsection{并发执行}
		make -j[n]\par
		j(jobs)表示多条任务同时进行,n是一个数字,表示同时并发数,比如make -j3。如果不带n将不对并发数进行限制。如果命令中有多个-j选项,则以最后一个为准。
	\subsubsection{进入子目录}
		make -C  dir\par
		make进入子目录dir执行。该命令对于组织大型项目很有效,称为递归make。例如项目主目录下有sub1,sub2两个目录。\par
		\begin{verbatim}
			subdir=sub1 sub2
			all:
			    for i in $(subdir) do;make -C $$(i) ;done ;
		\end{verbatim}\par
		\$\$(i)第意思是由于在make命令中出现了变量\$i,所以需要在该变量前面再加一个\$。	
	\section{扩展用法}
	\subsection{自动化建立依赖关系}
		对于c程序来说,目标依赖源程序,.c源程序中包括头文件。当改变.c源程序的时候,make时会重构程序,但是改变.h程序往往不能自动重构,这时候就需要建立.o文件与h头文件的依赖关系,从而当h文件改变的时候也能够重构程序。
		\par
		gcc编译器有一个-M选项,可以预编译c程序,建立.o文件与所依赖的.c文件以及.h文件之间的依赖关系。-MM选项可以建立.o文件与c文件,标准头文件以及用户h文件之间的依赖关系。
		\begin{verbatim}
			target=test
			objs=fun.o main.o
			depend_file=$(objs:%.o=%.d)
			-include $(depend_file) #包括进来依赖关系文件
			%.o:%.c
			    gcc $< -o $@
			%.d:%.c
			    gcc -M $< -o $@ #建立.d依赖关系文件,该文件中包括.o文件与.c,.h文件的依赖关系
			$(target):$(objs)
			    gcc $^ -o $@
			.PHONY:clean
			clean:
			    rm -f $(depend_file) $(objs) $(target)
		\end{verbatim}

	\subsection{条件判断}
	\begin{verbatim}
		ifeq(exp1,exp2)
		    command1
		else
		    command2
		endif
	\end{verbatim}
	举例:
	\begin{verbatim}
		CC?=gcc
		ifeq($(CC),cc)
		    @echo CC is cc 
		else
		    @echo CC is gcc
		endif
	\end{verbatim}
	\subsection{自动建立makefile}
		待续
\section{另外一种make工具cmake}
	待续 

\begin{thebibliography}{2}
	\bibitem{p1} man make,info make 
	\bibitem{p2} 徐海兵.GNU make中文手册. 2004,9
\end{thebibliography}

\newpage
\listoftables	
\newpage
\printindex	

\end{CJK}
\end{document}
